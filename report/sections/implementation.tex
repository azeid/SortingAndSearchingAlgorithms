We have made available the source code of all implementations at the repository \url{https://github.com/azeid/SortingAndSearchingAlgorithms}.

\subsection{Sorting Algorithms Correctness}
As part of adding the sorting algorithms for this project, we also added correctness checks to make sure that the sorting algorithm is producing a sorted result and that the resulting array is a permutation of the original input array. This we we have confidence that the algorithm is correct. Additionally, we added smaller unit tests for each algorithm to make sure we cover the corner cases and validate that the algorithms handle unexpected inputs gracefully.

\subsection{Test Cases}
In order for us to compare the time complexity of different sorting algorithms, we had to come up with different test cases. We made available all the test cases used in the comparison at the repository \url{https://github.com/azeid/SortingAndSearchingAlgorithms/tree/master/testCases}.\\
The naming convention used was "TestCase\textunderscore DataSize.txt", for example "testCases/SortedInAscendingOrderCase \textunderscore 1000.txt". The naming convention made it easy to identify test cases and parse the output report from the benchmark tool.

\subsubsection{Test Cases Generation}
Generating test cases is a tedious task if done manually; however, we automated the process of generating different test cases. We have 11 unique test cases that we ran on each algorithm. Additionally, we created the test case generation functions to generate any data size for any particular test case. Also, we tried to include test cases that cover the best, worst, and average cases for our sorting algorithms in order for the comparison to show advantages and disadvantages of different sorting algorithms based on test cases and data sizes.

\subsubsection{Test Cases Generation Correctness}
For every test case generator function we included unit tests to verify the correctness of the generated arrays. Given that the functions can handle any data size, we have to make sure that for large data sizes that correctness still holds. These unit tests gave us confidence that our test cases are what we expect them to be.

\subsubsection{Test Cases Used}
We have a total of 11 unique test cases. For each test case we created multiple data sizes starting from 100 to 10,000,000 (since 10 was excluded since it was too small for any comparison). We were not able to upload the 10,000,000 data size test cases as they were too large.

The test cases used are:
\begin{enumerate}
    \item Sorted In Ascending Order
    \item Sorted In Descending Order  
    \item Random Order  
    \item Random Order - High Numbers on First Half and Low Numbers of Second Half  
    \item Random Order - Low Numbers on First Half and High Numbers of Second Half  
    \item Sorted In Ascending Order - High Numbers on First Half and Low Numbers of Second Half 
    \item Sorted In Descending Order - High Numbers on First Half and Low Numbers of Second Half 
    \item Nearly Sorted In Ascending Order
    \item Nearly Sorted In Descending Order
    \item Same Value Array
    \item Merge Sort Worst Case (This was added since the Merge Sort worst case was not covered by our generic test cases)
    
\end{enumerate}

\subsubsection{Graphing Tool}

In order to represent the data from the benchmarks, we developed a graphing utility. The tool parses the resulting data files from the benchmarks, and it produces a graphical representation of the data. The data plotted is the execution times of the studied sorting algorithms for different input sizes and input cases. The utility has the ability to normalize the data using the results of any of the algorithms benchmarked. The data is organized by plotting the different input cases (the inputs cases mentioned in the previous section) in separate graphs. Each graph represents time of execution of the algorithms versus the size of the input data for an specific input case.

The tool was implemented using Python 3. It uses the matplotlib plotting library to draw the graphs. The source code of the utility is available in the repository of the project at the following URL: \url{https://github.com/azeid/SortingAndSearchingAlgorithms/tree/master/graph_generator}.

