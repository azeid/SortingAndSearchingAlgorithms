\documentclass[sigconf,table]{acmart}
\usepackage{graphics}
\usepackage[utf8]{inputenc}
\usepackage{algorithm}
\usepackage{algorithmic}
\usepackage{url}
\usepackage{setspace}
\usepackage{hyperref}
\usepackage{comment}

% Remove coyright and ACM reference of this paper
% We should probably remove this on the real paper
\settopmatter{printacmref=false}
\setcopyright{none}
\renewcommand\footnotetextcopyrightpermission[1]{}
\pagestyle{plain}
% --- end remove

\linespread{1.5}

\makeatletter
\def\subsubsection{\@startsection{subsubsection}{3}%
  \z@{.5\linespacing\@plus.7\linespacing}{.1\linespacing}%
  {\normalfont\itshape}}
\makeatother

\begin{document}
\title{Analyzing Different Sorting and Searching Algorithms}

\author{Zeid Ayssa}
\affiliation{
    \institution{University of Texas at Austin \\ Dept. of Electrical and Computer Engineering}
}
\email{azeid@utexas.edu}

\author{Jose Martinez Garcia-Vaso}
\affiliation{
    \institution{University of Texas at Austin \\ Dept. of Electrical and Computer Engineering}
}
\email{carlosgvaso@utexas.edu}

\author{Utkarsh Vardan}
\affiliation{
    \institution{University of Texas at Austin \\ Dept. of Electrical and Computer Engineering}
}
\email{uvardan@utexas.edu}

\begin{abstract}
In this project we will analyze a variety of sorting algorithms and explore how they compare to each other in terms of performance as well as time and space complexity. Additionally, we will also show how different sorting algorithms perform with different test cases and data-set sizes. Add to that, we will explore the possibility to visualize each sorting algorithm and demonstrate how effectually different data sets from different starting point can be sorted using different algorithms.

Some of the sorting algorithms that we are considering to study are:
\begin{itemize}
  \item Quick Sort
  \item Merge Sort
  \item Bubble Sort
  \item Heap Sort
  \item Shell Sort
  \item Radix Sort
  \item Tim Sort
\end{itemize}


\end{abstract}

\maketitle

\section{Introduction}
The sorting problem is one of the fundamental problems in Computer Science. It consists of obtaining a permutation of a sequence of numbers sorted in non-decreasing order. The problem is defined as follows\cite{clrs2009}:

\textbf{Input}: A sequence of $n$ numbers $\langle a_0, a_1, ..., a_{n-1} \rangle$.

\textbf{Output}: A permutation of the input sequence $\langle a_0', a_1', ..., a_{n-1}' \rangle$ such that $a_0' \leq a_1' \leq ... \leq a_{n-1}'$.

Over the times, there have been many algorithms developed to solve this problem. These algorithms employ a variety techniques, data structures and mathematical properties, which lead to some algorithms performing better than others for different input types. In our research, we focused specifically on the problem of sorting an array of integers, and we studied the performance of nine sorting algorithms when presented with multiple inputs of different sizes and makeups. We compared the performance of the following algorithms:

\begin{itemize}
  \item Bubble Sort.
  \item Insertion sort.
  \item Quicksort.
  \item Java Collections Framework Arrays sort implementation.
  \item Mergesort.
  \item Heapsort.
  \item Shell Sort.
  \item Radix Sort.
  \item Timsort.
\end{itemize}

In the next subsections, we will introduce the previous algorithms, and we will talk about their theoretical time and space complexities. We will also go over their best, average and worst case inputs.


\subsection{Bubble Sort}

Bubble sort works by passing the data in the collection by multiple passes. The data is processed from start to end, or left to right. Starting from the first value in the collection, the value is compared to next value, if the value is larger than the next value they are swapped. The largest value is moved to the right. This comparison and swap operation is repeated for each value in the collection until no swaps are performed. Then, the array is in sorted order.

\textit{Figure \ref{fig:bubblesort_ex}} shows how bubble sort works on an example array. In the 1st pass (see \textit{Figure \ref{fig:bubblesort_ex} (a)} and \textit{(b)}), we compare $4$ with $7$, and swap is not performed because $4$ is already smaller than $7$. Next, $7$ is compared and swapped with the next elements: $5$ and $5$ (\textit{(b)} and \textit{(c)}). Similarly in the third pass \textit{(d)}, $7$ is swapped with $6$. As shown in \textit{{e}}, the fourth pass \textit{(e)} yields a sorted array, since no swaps are performed. 

\textit{Figure \ref{fig:bubblesort}} shows the time and space complexities of Bubble sort.

\begin{figure}[!ht]
    \centering
    \begin{tabular}{|c|c|c|c|c|c|c|}
    \hline
    \cellcolor{gray}4 & \cellcolor{gray}7 & 5 & 5 & 6 & 8 & 9 \\
    \hline
    \multicolumn{7}{c}{(a)} \\
    \end{tabular}
    \,
    \begin{tabular}{|c|c|c|c|c|c|c|}
    \hline
    \cellcolor{lightgray}4 & \cellcolor{gray}5 & \cellcolor{gray}7 & 5 & 6 & 8 & 9 \\
    \hline
    \multicolumn{7}{c}{(b)} \\
    \end{tabular}
    \break
    \begin{tabular}{|c|c|c|c|c|c|c|}
    \hline
    \cellcolor{lightgray}4 & \cellcolor{lightgray}5 & \cellcolor{gray}5 & \cellcolor{gray}7 & 6 & 8 & 9 \\
    \hline
    \multicolumn{7}{c}{(c)} \\
    \end{tabular}
    \,
    \begin{tabular}{|c|c|c|c|c|c|c|}
    \hline
    \cellcolor{lightgray}4 & \cellcolor{lightgray}5 & \cellcolor{lightgray}5 & \cellcolor{gray}6 & \cellcolor{gray}7 & 8 & 9 \\
    \hline
    \multicolumn{7}{c}{(d)} \\
    \end{tabular}
    \break
    \begin{tabular}{|c|c|c|c|c|c|c|}
    \hline
    \cellcolor{lightgray}4 & \cellcolor{lightgray}5 & \cellcolor{lightgray}5 & \cellcolor{lightgray}6 & \cellcolor{lightgray}7 & \cellcolor{lightgray}8 & \cellcolor{lightgray}9 \\
    \hline
    \multicolumn{7}{c}{(e)} \\
    \end{tabular}
    
    \caption{Bubble sort visualization. (a) shows the input array. The comparisons are shown in dark gray, and sorted elements are shown in light gray. (b) shows pass 1 result, (c) shows pass 2 result, (d) shows pass 3 result and (e) shows pass 4 result which returns the array in sorted order.}
    \label{fig:bubblesort_ex}
\end{figure}

\begin{figure}[!ht]
    \centering
    \begin{tabular}{l|l}
    \multicolumn{2}{c}{\textbf{Time Complexity}} \\
    \hline
    Best Case    & $\Omega(n)$ \\
    Average Case & $\Theta(n^2)$ \\
    Worst Case   & $\Theta(n^2)$ \\
    \end{tabular}
    \quad\quad
    \begin{tabular}{l|l}
    \multicolumn{2}{c}{\textbf{Space Complexity}} \\
    \hline
    Worst Case   & $O(1)$
    \end{tabular}
    
    \caption{Bubble sort time\cite{kazim2017} and space complexities\cite{big-o}.}
    \label{fig:bubblesort}
\end{figure}

\subsection{Insertion Sort}

The insertion sort algorithm sorts the the array of integers in-place in a similar way as we would sort a had of cards. It divides the array in a left and right sides, with the goal of ending with all the numbers sorted in the left side and no numbers in the right side. We start with all the numbers in the right side, and we add them one by one to the left side in ascending order. To make sure the numbers are inserted in the correct place, the numbers are compared from right to left with the numbers already in the left side of the array\cite{clrs2009}.

Insertion performs single pass in the collection of data. All the data on the left side of the item currently been evaluated is know to be sorted and all the data to the right is considered to be unsorted. \textit{Figure \ref{fig:insertionsort_ex}} shows insertion sort in action. We start with an unsorted array (see \textit{Figure \ref{fig:insertionsort_ex} (a)}). The first item in the collection, since there is nothing to the left of $6$, it is considered to be sorted. Since $4$ is less than $6$, we make a swap, and $4$ and $6$ are considered to be sorted. The rest of the numbers are unsorted. Since $5$ is less than $6$, we perform a swap between $5$ and $6$. We continue this operation for $9$, since $9$ is greater than $6$, no swap is performed. Since $7$ is less than $9$, we swap $7$ and $9$. Finally, since $8$ is less than $9$, we again swap $8$ and $9$. In a single pass, we have sorted the entire array of data using the insertion sort (see \textit{Figure \ref{fig:insertionsort_ex} (b)}).

\textit{Figure \ref{fig:insertionsort}} shows the time and space complexities of insertion sort.

\begin{figure}[!ht]
    \centering
    \begin{tabular}{|c|c|c|c|c|c|c|}
    \hline
    6 & 4 & 5 & 5 & 9 & 7 & 8 \\
    \hline
    \multicolumn{7}{c}{(a)} \\
    \end{tabular}
    \quad\quad
    \begin{tabular}{|c|c|c|c|c|c|c|}
    \hline
    4 & 5 & 5 & 6 & 7 & 8 & 9 \\
    \hline
    \multicolumn{7}{c}{(b)} \\
    \end{tabular}
    
    \caption{Insertion sort visualization. (a) shows the input array. (b) shows the array after one pass of insertion sort.}
    \label{fig:insertionsort_ex}
\end{figure}

\begin{figure}[!ht]
    \centering
    \begin{tabular}{l|l}
    \multicolumn{2}{c}{\textbf{Time Complexity}} \\
    \hline
    Best Case    & $\Theta(n)$ \\
    Average Case & $\Theta(n^2)$ \\
    Worst Case   & $\Theta(n^2)$ \\
    \end{tabular}
    \quad\quad
    \begin{tabular}{l|l}
    \multicolumn{2}{c}{\textbf{Space Complexity}} \\
    \hline
    Worst Case   & $O(1)$ \\
    \end{tabular}
    
    \caption{Insertion sort time\cite{clrs2009} and space complexities\cite{big-o}.}
    \label{fig:insertionsort}
\end{figure}


\subsection{Quicksort}

Quick sort is a divide and conquer algorithm. It’s also one of the most commonly used general purpose sorting algorithm in computer science. Since it is a divide and conquer algorithm, we will be dividing the data into smaller sets. In quick sort, the arrays are not necessarily split in half, rather a pivot value is picked based on the rules or a heuristic. Once a pivot value is picked, all the items in the array smaller than the pivot are placed at the left side of pivot, and entries to the right of the pivot are larger than the pivot. This pivot and partition operation is performed repeatedly on left and right side of partitions until all the items are sorted. 

In the unsorted array of data shown in \textit{Figure \ref{fig:quicksort_ex} (a)}, we choose $5$ as the pivot. All the entries less than $5$ will be moved to the left of $5$, and values greater than $5$ will be moved to the right of $5$. Therefore, $4$ and $8$ are swapped resulting in \textit{Figure \ref{fig:insertionsort_ex} (b)}. Next we choose $2$ as the next partition, all the values to the left of $2$ are larger, and values to the right are smaller. Therefore, we need to pull $2$ out of the array, and place all the values around in their appropriate location as shown in \textit{Figure \ref{fig:quicksort_ex} (c)}. If we pick $3$ as pivot, entries to the left of $3$ are smaller. Therefore, it is sorted. Similarly, we repeatedly pick to pivot to the right side of $5$ until all elements are sorted as displayed in \textit{Figure \ref{fig:quicksort_ex} (d)}.

\textbf{Figure \ref{fig:quicksort}} shows the time and space complexities of insertion sort.

\begin{figure}[!ht]
    \centering
    \begin{tabular}{|c|c|c|c|c|c|c|c|}
    \hline
    3 & 8 & 2 & 1 & \cellcolor{lightgray}5 & 4 & 6 & 7 \\
    \hline
    \multicolumn{8}{c}{(a)} \\
    \end{tabular}
    \,
    \begin{tabular}{|c|c|c|c|c|c|c|c|}
    \hline
    3 & 4 & \cellcolor{lightgray}2 & 1 & 5 & 8 & 6 & 7 \\
    \hline
    \multicolumn{8}{c}{(b)} \\
    \end{tabular}
    \break
    \begin{tabular}{|c|c|c|c|c|c|c|c|}
    \hline
    1 & 2 & \cellcolor{lightgray}3 & 4 & 5 & 7 & 6 & 8 \\
    \hline
    \multicolumn{8}{c}{(c)} \\
    \end{tabular}
    \,
    \begin{tabular}{|c|c|c|c|c|c|c|c|}
    \hline
    1 & 2 & 3 & 4 & 5 & 6 & 7 & 8 \\
    \hline
    \multicolumn{8}{c}{(d)} \\
    \end{tabular}
    
    \caption{Quicksort visualization. (a) shows the input array. (b) shows the array after using $5$ as the pivot. (c) shows the array after uisng 3 as the pivot. (d) shows the sorted array.}
    \label{fig:quicksort_ex}
\end{figure}

\begin{figure}[!ht]
    \centering
    \begin{tabular}{l|l}
    \multicolumn{2}{c}{\textbf{Time Complexity}} \\
    \hline
    Best Case    & $\Theta(n \, \lg n)$ \\
    Average Case & $\Theta(n \, \lg n)$ \\
    Worst Case   & $\Theta(n^2)$ \\
    \end{tabular}
    \quad\quad
    \begin{tabular}{l|l}
    \multicolumn{2}{c}{\textbf{Space Complexity}} \\
    \hline
    Worst Case   & $O(\lg n)$
    \end{tabular}
    
    \caption{Quicksort time\cite{clrs2009} and space complexities\cite{big-o}.}
    \label{fig:quicksort}
\end{figure}


\subsection{Mergesort}

Merge sort works by recursively splitting the data in half. For example, an array of $10$ items would be split in the middle into two sub-arrays of $5$ items each. The splitting continues until each sub-array has only one item in it. Since each sub-array has only one item in it, that sub-array is known to be sorted. At this point, the sub-arrays are merged, but the values are put together in sorted order. After each merger, the sorted sub-array doubles in size, and this procedure continues until all the sub-arrays are merged and fully sorted.

Initially, the arrays are recursively split in half. First, the initial array (see \textit{Figure \ref{fig:mergesort_ex} (a)}) is split into $2$ sub-arrays of $4$ entries each as shown in \textit{Figure \ref{fig:mergesort_ex} (b)}. Next, the sub-arrays are split in arrays of $2$ entries each as shown in \textit{Figure \ref{fig:mergesort_ex} (c)}. Then, the resulting sub-arrays are split into $8$ sub-arrays of $1$ entry each as shown in \textit{Figure \ref{fig:mergesort_ex} (d)}. Because there is only one item in the sub-arrays shown in \textit{Figure \ref{fig:mergesort_ex} (d)}, the entries in the sub-arrays are sorted within their sub-array. For the reconstruction phase, we merge each single entry sub-array back to sub-arrays of $2$ entries each by sorting them. Then, $4$ and $9$ are reconstructed into a sub-array as they are already sorted. $3$ and $2$ are reconstructed into an array with $2$ and $3$ in sorted order. $6$ and $5$ are reconstructed as $5$ and $6$. $7$ and $8$ are reconstructed in the same order. The result is shown in \textit{Figure \ref{fig:mergesort_ex} (e)}. Similarly, we merge the resulting sub-arrays of $2$ entries each back to $2$ sub-arrays of $4$ entries each in sorted order, which is shown in \textit{Figure \ref{fig:mergesort_ex} (f)}. In the final reconstruction step, the $2$ remaining sub-arrays are merged into a single sorted array as shown in \textit{Figure \ref{fig:mergesort_ex} (g)}.

\textit{Figure \ref{fig:mergesort}} shows the time and space complexities of Mergesort.

\begin{figure}[!ht]
    \centering
    \begin{tabular}{|c|c|c|c|c|c|c|c|}
    \hline
    4 & 9 & 3 & 2 & 6 & 5 & 7 & 8 \\
    \hline
    \multicolumn{8}{c}{(a)} \\
    \end{tabular}
    \,
    \begin{tabular}{|c|c|c|c|c|c|c|c|}
    \hline
    4 & 9 & 3 & 2 & \cellcolor{lightgray}6 & \cellcolor{lightgray}5 & \cellcolor{lightgray}7 & \cellcolor{lightgray}8 \\
    \hline
    \multicolumn{8}{c}{(b)} \\
    \end{tabular}
    \break
    \begin{tabular}{|c|c|c|c|c|c|c|c|}
    \hline
    4 & 9 & \cellcolor{lightgray}3 & \cellcolor{lightgray}2 & 6 & 5 & \cellcolor{lightgray}7 & \cellcolor{lightgray}8 \\
    \hline
    \multicolumn{8}{c}{(c)} \\
    \end{tabular}
    \,
    \begin{tabular}{|c|c|c|c|c|c|c|c|}
    \hline
    4 & \cellcolor{lightgray}9 & 3 & \cellcolor{lightgray}2 & 6 & \cellcolor{lightgray}5 & 7 & \cellcolor{lightgray}8 \\
    \hline
    \multicolumn{8}{c}{(d)} \\
    \end{tabular}
    \break
    \begin{tabular}{|c|c|c|c|c|c|c|c|}
    \hline
    4 & 9 & \cellcolor{lightgray}2 & \cellcolor{lightgray}3 & 5 & 6 & \cellcolor{lightgray}7 & \cellcolor{lightgray}8 \\
    \hline
    \multicolumn{8}{c}{(e)} \\
    \end{tabular}
    \,
    \begin{tabular}{|c|c|c|c|c|c|c|c|}
    \hline
    2 & 3 & 4 & 9 & \cellcolor{lightgray}5 & \cellcolor{lightgray}6 & \cellcolor{lightgray}7 & \cellcolor{lightgray}8 \\
    \hline
    \multicolumn{8}{c}{(f)} \\
    \end{tabular}
    \break
    \begin{tabular}{|c|c|c|c|c|c|c|c|}
    \hline
    2 & 3 & 4 & 5 & 6 & 7 & 8 & 9 \\
    \hline
    \multicolumn{8}{c}{(g)} \\
    \end{tabular}
    
    \caption{Mergesort visualization. (a) shows the input array. (b), (c) and (d) show how mergesort splits the initial array into sub-arrays. Different sub-arrays are distinguished by the different cell backgrounds. (e), (f) and (g) shows how mergesort joins back the sub-arrays in sorted order.}
    \label{fig:mergesort_ex}
\end{figure}

\begin{figure}[!ht]
    \centering
    \begin{tabular}{l|l}
    \multicolumn{2}{c}{\textbf{Time Complexity}} \\
    \hline
    Best Case    & $\Theta(n \, \lg n)$ \\
    Average Case & $\Theta(n \, \lg n)$ \\
    Worst Case   & $\Theta(n \, \lg n)$ \\
    \end{tabular}
    \quad\quad
    \begin{tabular}{l|l}
    \multicolumn{2}{c}{\textbf{Space Complexity}} \\
    \hline
    Worst Case   & $O(n)$
    \end{tabular}
    
    \caption{Mergesort time\cite{clrs2009} and space complexities\cite{big-o}.}
    \label{fig:mergesort}
\end{figure}


\subsection{Heapsort}

Heapsort is comparison based sorting technique based on binary heap data structure. It is similar to selection sort in terms of finding the maximum element, and placing the maximum element in the end. This process is repeated for the remaining elements in an array. 

Initially, heap data structure is created using a min-heap or max-heap for the unsorted list of elements. The first element (root node) of the heap is either largest or smallest depending if it is a min or max-heap. We take out first element of the heap, and we place it at the end of the array. Then we reduce the heap size to keep the sorted element outside of the heap. We again make the heap using the remaining elements, and we pick the first element of the heap store in the end of the array. This process is repeated until we have the array completely sorted. 

Consider the example shown in \textit{Figure \ref{fig:heapsort_ex}}, where we use a max-heap. After building the max-heap, we get the array shown in \textit{Figure \ref{fig:heapsort_ex}(b)}. Since $9$ is the largest element (root node of the heap), it is removed from the heap, and it is swapped with the last entry of the array as shown in \textit{(c)}. The size of the heap is reduced by one, which is shown by graying out the entries that are no longer part of the heap. Rebuilding the max-heap with the remaining entries, we obtain the array shown in \textit{(d)}. Since $6$ is the root of the heap, it is swapped with the last element of the heap, and it is removed from the heap by reducing its size by one (see \textit{(e)}). \textit{(f)} shows the result of rebuilding the max-heap, and \textit{(g)} shows the result of removing the root node of the heap. Continuing this process, we reach a fully sorted array as shown in \textit{Figure \ref{fig:heapsort_ex} (j)}.

\textit{Figure \ref{fig:heapsort}} shows the time and space complexities of Heapsort.

\begin{figure}[!ht]
    \centering
    \begin{tabular}{|c|c|c|c|c|}
    \hline
    5 & 9 & 4 & 6 & 3 \\
    \hline
    \multicolumn{5}{c}{(a)} \\
    \end{tabular}
    \quad\quad\quad
    \begin{tabular}{|c|c|c|c|c|}
    \hline
    9 & 6 & 4 & 5 & 3 \\
    \hline
    \multicolumn{5}{c}{(b)} \\
    \end{tabular}
    \break
    \begin{tabular}{|c|c|c|c|c|}
    \hline
    3 & 6 & 4 & 5 & \cellcolor{lightgray}9 \\
    \hline
    \multicolumn{5}{c}{(c)} \\
    \end{tabular}
    \quad\quad\quad
    \begin{tabular}{|c|c|c|c|c|}
    \hline
    6 & 5 & 4 & 3 & \cellcolor{lightgray}9 \\
    \hline
    \multicolumn{5}{c}{(d)} \\
    \end{tabular}
    \break
    \begin{tabular}{|c|c|c|c|c|}
    \hline
    3 & 5 & 4 & \cellcolor{lightgray}6 & \cellcolor{lightgray}9 \\
    \hline
    \multicolumn{5}{c}{(e)} \\
    \end{tabular}
    \quad\quad\quad
    \begin{tabular}{|c|c|c|c|c|}
    \hline
    5 & 4 & 3 & \cellcolor{lightgray}6 & \cellcolor{lightgray}9 \\
    \hline
    \multicolumn{5}{c}{(f)} \\
    \end{tabular}
    \break
    \begin{tabular}{|c|c|c|c|c|}
    \hline
    3 & 4 & \cellcolor{lightgray}5 & \cellcolor{lightgray}6 & \cellcolor{lightgray}9 \\
    \hline
    \multicolumn{5}{c}{(g)} \\
    \end{tabular}
    \quad\quad\quad
    \begin{tabular}{|c|c|c|c|c|}
    \hline
    4 & 3 & \cellcolor{lightgray}5 & \cellcolor{lightgray}6 & \cellcolor{lightgray}9 \\
    \hline
    \multicolumn{5}{c}{(h)} \\
    \end{tabular}
    \break
    \begin{tabular}{|c|c|c|c|c|}
    \hline
    3 & \cellcolor{lightgray}4 & \cellcolor{lightgray}5 & \cellcolor{lightgray}6 & \cellcolor{lightgray}9 \\
    \hline
    \multicolumn{5}{c}{(i)} \\
    \end{tabular}
    \quad\quad\quad
    \begin{tabular}{|c|c|c|c|c|}
    \hline
    \cellcolor{lightgray}3 & \cellcolor{lightgray}4 & \cellcolor{lightgray}5 & \cellcolor{lightgray}6 & \cellcolor{lightgray}9 \\
    \hline
    \multicolumn{5}{c}{(j)} \\
    \end{tabular}
    
    \caption{Heapsort visualization. (a) shows the input array. (b),  shows how initial array is made into a max-heap. In (c), the root node is swapped with the last element of the array, and the heap size is reduced by one. The elements in the heap are in white, while the elements outside the heap are greyed out. The previous steps are repeated from (d) to (j), until a fully sorted array is produced as shown in (j)}
    \label{fig:heapsort_ex}
\end{figure}

\begin{figure}[!ht]
    \centering
    \begin{tabular}{l|l}
    \multicolumn{2}{c}{\textbf{Time Complexity}} \\
    \hline
    Best Case    & $\Theta(n \, \lg n)$ \\
    Average Case & $\Theta(n \, \lg n)$ \\
    Worst Case   & $\Theta(n \, \lg n)$ \\
    \end{tabular}
    \quad\quad
    \begin{tabular}{l|l}
    \multicolumn{2}{c}{\textbf{Space Complexity}} \\
    \hline
    Worst Case   & $O(1)$
    \end{tabular}
    
    \caption{Heapsort time\cite{clrs2009} and space complexities\cite{big-o}.}
    \label{fig:heapsort}
\end{figure}

\subsection{Shell Sort}

Shell sort is a generalization of the insertion sort algorithm. Using shell sort, we compare the elements of the array that are far apart, rather than adjacent. For this purpose, we define the variable $gap$, and we compare the elements that are $gap$ number of elements away from each other. This divides the initial array in a $gap$ number of interleaved sub-arrays, whose elements are sorted by comparison. With every iteration, we reduce the gap between the elements being compared, and we sort the sub-arrays. When we sort in the last pass where the $gap = 1$, the array is fully sorted. In the last pass, shell sort behaves like insertion sort.

Consider the example in \textit{Figure \ref{fig:shellsort_ex} (a)}, where we want to sort the array in ascending order. In this example, we start with a value of $gap = 2$, but the initial value of $gap$ could be anything, and we reduce the value of gap by dividing it by $2$ with each pass. In \textit{(b)}, we show how the initial array is divided in interleaved sub-arrays which are shown in different colors. Next, we sort the sub-arrays individually, as shown in \textit{(c)} and \textit{(d)}. Now, we are done with the iteration, and we can move to the next iteration with $gap = 2/1 = 1$. This means the whole array will be considered now as shown in \textit{(e)}. Now, we would rearrange the elements in sorted order, but the items were already sorted in our example (see \textit{(f)}).

\textit{Figure \ref{fig:shellsort}} shows the time and space complexities of Shell sort.

\begin{figure}[!ht]
    \centering
    \begin{tabular}{|c|c|c|c|c|}
    \hline
    13 & 35 & 55 & 3 & 4 \\
    \hline
    \multicolumn{5}{c}{(a)} \\
    \end{tabular}
    \quad\quad\quad
    \begin{tabular}{|c|c|c|c|c|}
    \hline
    \cellcolor{lightgray}13 & 35 & \cellcolor{lightgray}55 & 3 & \cellcolor{lightgray}4 \\
    \hline
    \multicolumn{5}{c}{(b)} \\
    \end{tabular}
    \break
    \begin{tabular}{|c|c|c|c|c|}
    \hline
    \cellcolor{lightgray}4 & 35 & \cellcolor{lightgray}13 & 3 & \cellcolor{lightgray}55 \\
    \hline
    \multicolumn{5}{c}{(c)} \\
    \end{tabular}
    \quad\quad\quad
    \begin{tabular}{|c|c|c|c|c|}
    \hline
    \cellcolor{lightgray}4 & 3 & \cellcolor{lightgray}13 & 35 & \cellcolor{lightgray}55 \\
    \hline
    \multicolumn{5}{c}{(d)} \\
    \end{tabular}
    \break
    \begin{tabular}{|c|c|c|c|c|}
    \hline
    \cellcolor{lightgray}4 & \cellcolor{lightgray}3 & \cellcolor{lightgray}13 & \cellcolor{lightgray}35 & \cellcolor{lightgray}55 \\
    \hline
    \multicolumn{5}{c}{(e)} \\
    \end{tabular}
    \quad\quad\quad
    \begin{tabular}{|c|c|c|c|c|}
    \hline
    \cellcolor{lightgray}4 & \cellcolor{lightgray}3 & \cellcolor{lightgray}13 & \cellcolor{lightgray}35 & \cellcolor{lightgray}55 \\
    \hline
    \multicolumn{5}{c}{(f)} \\
    \end{tabular}
    
    \caption{Shell sort visualization. (a) shows the input array. (b), (c) and (d) show how mergesort splits the initial array into sub-arrays. (e), (f) and (g) shows how mergesort joins back the sub-arrays in sorted order.}
    \label{fig:shellsort_ex}
\end{figure}

\begin{figure}[!ht]
    \centering
    \begin{tabular}{l|l}
    \multicolumn{2}{c}{\textbf{Time Complexity}} \\
    \hline
    Best Case    & $\Omega(n \, \lg n)$ \\
    Average Case & $\Theta(n \, (\lg n)^2)$ \\
    Worst Case   & $\Theta(n \, (\lg n)^2)$ \\
    \end{tabular}
    \quad\quad
    \begin{tabular}{l|l}
    \multicolumn{2}{c}{\textbf{Space Complexity}} \\
    \hline
    Worst Case   & $O(1)$
    \end{tabular}
    
    \caption{Shell sort time\cite{big-o} and space complexities\cite{big-o}.}
    \label{fig:shellsort}
\end{figure}


\subsection{Radix Sort}

Radix sort is not a comparative integer sorting algorithm as the other algorithms that we have seen so far. It instead sorts data lexicographically. In the case that the data is made of integers, Radix sort uses the digits of the integers to sort the data, and it uses Counting sort to help with sorting.

Consider the input array shown in \textit{Figure \ref{fig:radixsort_ex} (a)}. In order to sort elements using radix sort, we use Counting sort using the last digit of each number as the sorting keys (see \textit{(b)}). Sorting the elements, we obtain the array shown in \textit{(c)}. As counting sort is stable sorting algorithm, the elements with the same key will appear in the same order. In the next step, we sort the elements by counting sort using the second to last digit as key (see \textit{(d)}), and we obtain the array shown in \textit{(e)}. Repeating the same procedure with the first digit as key (see \textit{(f)}), we notice that there are few numbers without any number in the first place. In this case, we pad the numbers zeroes as their key. Once we sort the first digit using counting sort, the whole array is sorted as displayed in \textit{(g)}.

\textit{Figure \ref{fig:radixsort}} shows the time and space complexities of Radix sort.

\begin{figure}[!ht]
    \centering
    \begin{tabular}{|c|c|c|c|c|c|}
    \hline
    053 & 089 & 150 & 036 & 633 & 233 \\
    \hline
    \multicolumn{6}{c}{(a)} \\
    \end{tabular}
    \,
    \begin{tabular}{|c|c|c|c|c|c|}
    \hline
    05\textbf{3} & 08\textbf{9} & 15\textbf{0} & 03\textbf{6} & 63\textbf{3} & 23\textbf{3} \\
    \hline
    \multicolumn{6}{c}{(b)} \\
    \end{tabular}
    \break
    \begin{tabular}{|c|c|c|c|c|c|}
    \hline
    15\textbf{0} & 05\textbf{3} & 63\textbf{3} & 23\textbf{3} & 03\textbf{6} & 08\textbf{9} \\
    \hline
    \multicolumn{6}{c}{(c)} \\
    \end{tabular}
    \,
    \begin{tabular}{|c|c|c|c|c|c|}
    \hline
    1\textbf{5}0 & 0\textbf{5}3 & 6\textbf{3}3 & 2\textbf{3}3 & 0\textbf{3}6 & 0\textbf{8}9 \\
    \hline
    \multicolumn{6}{c}{(d)} \\
    \end{tabular}
    \break
    \begin{tabular}{|c|c|c|c|c|c|}
    \hline
    6\textbf{3}3 & 2\textbf{3}3 & 0\textbf{3}6 & 1\textbf{5}0 & 0\textbf{5}3 & 0\textbf{8}9 \\
    \hline
    \multicolumn{6}{c}{(e)} \\
    \end{tabular}
    \,
    \begin{tabular}{|c|c|c|c|c|c|}
    \hline
    \textbf{6}33 & \textbf{2}33 & \textbf{0}36 & \textbf{1}50 & \textbf{0}53 & \textbf{0}89 \\
    \hline
    \multicolumn{6}{c}{(f)} \\
    \end{tabular}
    \break
    \begin{tabular}{|c|c|c|c|c|c|}
    \hline
    \textbf{0}36 & \textbf{0}53 & \textbf{0}89 & \textbf{1}50 & \textbf{2}33 & \textbf{6}33  \\
    \hline
    \multicolumn{6}{c}{(g)} \\
    \end{tabular}
    
    \caption{Radix sort visualization. (a) shows the input array. In (b), the sorting key is marked in boldface, and the resulting array after sorting is shown in (c). This procedure is repeated in (d)-(e) and (f)-(g). Notice the numbers have been padded with zeroes on the left, so that they can be sorted.}
    \label{fig:radixsort_ex}
\end{figure}

\begin{figure}[!ht]
    \centering
    \begin{tabular}{l|l}
    \multicolumn{2}{c}{\textbf{Time Complexity}} \\
    \hline
    Best Case    & $\Omega(n k)$ \\
    Average Case & $\Theta(n k)$ \\
    Worst Case   & $\Theta(n k)$ \\
    \end{tabular}
    \quad\quad
    \begin{tabular}{l|l}
    \multicolumn{2}{c}{\textbf{Space Complexity}} \\
    \hline
    Worst Case   & $O(n + k)$
    \end{tabular}
    
    \caption{Radix sort time\cite{kazim2017} and space complexities\cite{big-o}.}
    \label{fig:radixsort}
\end{figure}


\subsection{Timsort}

Timsort is a hybrid sorting algorithm. Timsort is a combination of Insertion sort and Mergesort. We basically divide an array into sub-arrays of length $RUN$, where $RUN$ is constant. Next, we apply binary insertion sort on each sub-array. Then, we merge the sub-arrays using Mergesort recursively to get the resulting sorted array. Timsort is extremely fast for nearly sorted data sequence. Timsort is also the default sorting algorithm in Java and Python, and it was implemented in 2002 by Tim Peter. 

Consider the example shown in \textit{Figure \ref{fig:timsort_ex}}, where an array of $10$ elements is sorted using $RUN=5$. \textit{{b}} shows how the initial array \textit{(a)} is divided into sub-arrays of size $RUN$. Using Insertion sort, we sort the sub-arrays as shown in \textit{(c)}. Finally, the sub-arrays are merged back in sorted order using Mergesort. The resulting sorted array is shown in \textit{(d)}.

\textit{Figure \ref{fig:timsort}} shows the time and space complexities of Timsort.

\begin{figure}[!ht]
    \centering
    \begin{tabular}{|c|c|c|c|c|c|c|c|c|c|}
    \hline
    3 & 8 & 35 & 30 & 23 & 10 & 40 & 50 & 55 & 52 \\
    \hline
    \multicolumn{10}{c}{(a)} \\
    \end{tabular}
    \break
    \begin{tabular}{|c|c|c|c|c|c|c|c|c|c|}
    \hline
    3 & 8 & 35 & 30 & 23 & \cellcolor{lightgray}10 & \cellcolor{lightgray}40 & \cellcolor{lightgray}50 & \cellcolor{lightgray}55 & \cellcolor{lightgray}52 \\
    \hline
    \multicolumn{10}{c}{(b)} \\
    \end{tabular}
    \break
    \begin{tabular}{|c|c|c|c|c|c|c|c|c|c|}
    \hline
    3 & 8 & 23 & 30 & 35 & \cellcolor{lightgray}10 & \cellcolor{lightgray}40 & \cellcolor{lightgray}50 & \cellcolor{lightgray}52 & \cellcolor{lightgray}55 \\
    \hline
    \multicolumn{10}{c}{(c)} \\
    \end{tabular}
    \break
    \begin{tabular}{|c|c|c|c|c|c|c|c|c|c|}
    \hline
    \cellcolor{lightgray}3 & \cellcolor{lightgray}8 & \cellcolor{lightgray}10 & \cellcolor{lightgray}23 & \cellcolor{lightgray}30 & \cellcolor{lightgray}35 & \cellcolor{lightgray}40 & \cellcolor{lightgray}50 & \cellcolor{lightgray}52 & \cellcolor{lightgray}55 \\
    \hline
    \multicolumn{10}{c}{(d)} \\
    \end{tabular}
    
    \caption{Timsort visualization. (a) shows the input array. (b) shows the array being split into sub-arrays of size $RUN$. (c) shows the sub-arrays after being sorted using Insertion sort. (d) shows the resulting sorted array after using Mergesort join back the sub-arrays in sorted order.}
    \label{fig:timsort_ex}
\end{figure}

\begin{figure}[!ht]
    \centering
    \begin{tabular}{l|l}
    \multicolumn{2}{c}{\textbf{Time Complexity}} \\
    \hline
    Best Case    & $\Omega(n)$ \\
    Average Case & $\Theta(n \, \lg n)$ \\
    Worst Case   & $\Theta(n \, \lg n)$ \\
    \end{tabular}
    \quad\quad
    \begin{tabular}{l|l}
    \multicolumn{2}{c}{\textbf{Space Complexity}} \\
    \hline
    Worst Case   & $O(n)$
    \end{tabular}
    
    \caption{Timsort time\cite{ajnp2017} and space complexities\cite{big-o}.}
    \label{fig:timsort}
\end{figure}


% Temporarily commented out for the proposal

\section{Implementation}
The sorting algorithms and the benchmark tool are written in Java using JDk 11. We have made available the source code of all implementations at the public GitHub repository \url{https://github.com/azeid/SortingAndSearchingAlgorithms}.

\subsection{Sorting Algorithms Correctness}
As part of adding the sorting algorithms for this project, we also added correctness checks to make sure that the sorting algorithm is producing a sorted result and that the resulting array is a permutation of the original input array. This we we have confidence that the algorithm is correct. Additionally, we added smaller unit tests for each algorithm to make sure we cover the corner cases and validate that the algorithms handle unexpected inputs gracefully.

\subsection{Test Cases}
In order for us to compare the time complexity of different sorting algorithms, we had to come up with different test cases. We made available all the test cases used in the comparison at the repository \url{https://github.com/azeid/SortingAndSearchingAlgorithms/tree/master/testCases}.\\
The naming convention used for test files was "TestCase\textunderscore DataSize.txt". For example, in the case of sorted in ascending order with 1000 elements the file name would be "testCases/SortedInAscendingOrderCase \textunderscore 1000.txt". This naming convention made it easy for us to identify test cases and parse the output report from the benchmark tool.

\subsubsection{Test Cases Generation}
Generating test cases is a tedious task if done manually; however, we automated the process of generating different test cases. We have 11 unique test cases that we ran on each algorithm. Additionally, we created the test case generation functions to generate any data size for any particular test case. Also, we tried to include test cases that cover the best, worst, and average cases for our sorting algorithms in order for the comparison to show advantages and disadvantages of different sorting algorithms based on test cases and data sizes.

\subsubsection{Test Cases Generation Correctness}
For every test case generator function we implemented unit tests to verify the correctness of the generated arrays. Given that the functions can handle any data size, we had to make sure that for large data sizes that correctness still holds. These unit tests gave us confidence that our test cases are what we expect them to be.

\subsubsection{Test Cases Included in Results}
We have a total of 11 unique test cases. For each test case we created multiple data sizes starting from 100 to 10,000,000 (10 was excluded since it was too small for any comparison). We were not able to upload the 10,000,000 data size test cases as the files were too large.

The test cases used are:
\begin{enumerate}
    \item Sorted In Ascending Order
    \item Sorted In Descending Order  
    \item Random Order  
    \item Random Order - High Numbers on First Half and Low Numbers on Second Half  
    \item Random Order - Low Numbers on First Half and High Numbers on Second Half  
    \item Sorted In Ascending Order - High Numbers on First Half and Low Numbers on Second Half 
    \item Sorted In Descending Order - High Numbers on First Half and Low Numbers on Second Half 
    \item Nearly Sorted In Ascending Order
    \item Nearly Sorted In Descending Order
    \item Same Value Array
    \item Merge Sort Worst Case (This was added since the Merge Sort worst case was not covered by our generic test cases)
    
\end{enumerate}

\subsubsection{Graphing Tool}

In order to represent the data from the benchmarks, we developed a graphing utility. The tool parses the resulting data files from the benchmarks, and it produces a graphical representation of the data. The data plotted is the execution times of the studied sorting algorithms for different input sizes and input cases. The utility has the ability to normalize the data using the results of any of the algorithms benchmarked. The data is organized by plotting the different input cases (the inputs cases mentioned in the previous section) in separate graphs. Each graph represents time of execution of the algorithms versus the size of the input data for an specific input case.

The tool was implemented using Python 3. It uses the matplotlib plotting library to draw the graphs. The source code of the utility is available in the repository of the project at the following URL: \url{https://github.com/azeid/SortingAndSearchingAlgorithms/tree/master/graph_generator}.



\section{Evaluation}

\subsection{Methodology}

All of our implementations were written in Java. Initially we thought of measuring the performance of our implementations with a naive approach where we would measure the time either in milliseconds or nanoseconds for sorting the input array. We could for example simply use \texttt{System.currentTimeMillis()} or \texttt{System.nanoTime()}. However we quickly found out how unreliable the results would be, since we noticed really inconsistent results between runs. As Ponge\cite{architectBenchmarking} explains, this kind of benchmarking might be viable in programs written in statically compiled languages like C. However Java runs on a Virtual Machine and it uses \emph{Just-in-time} compilation, so the first time the code is run it is actually being interpreted and then is compiled to native code, depending on the actual platform that is running. Furthermore, the VM tries to use all kinds of different optimization like loop unrolling, inlining functions or on-stack replacements, making it difficult to get consistent results. \\
We decided to use Java Microbenchmark Harness (JMH)\footnote{http://openjdk.java.net/projects/code-tools/jmh/} for measuring the performance of our implementations. JMH is an open source benchmarking tool part of the OpenJDK. Although it does not entirely prevent all common pitfalls and inconsistencies introduced by the JVM, it does help mitigating them. \\
Next step was to use the generated test cases files as inputs for our benchmarks. There are different modes to run benchmarks in JMH. We decided to measure the average time of an operation in microseconds, where an operation is sorting the array for any given algorithm and input array. In this mode, JMH considers an iteration to be a slice of time running as many operations as possible, it measures the time for each operation and averages it. In order to avoid some of the JIT inconsistencies and other JVM  optimizations, JMH runs a few warm-up iterations. After that it runs, by default, 5 iterations where the results are actually recorded. For our measuring purposes we decided to run 3  five seconds warm-up iterations and 5 ten seconds actual iterations.\\

The overall benchmark running time was over 36 hours. This was because we had large data sizes and due to the bad performance of some sorting algorithms in their worst case such as Bubble sort. We had to exclude Insertion and Bubble sort when running the test cases with a data size of 10,000,000.

\subsection{Results}

TODO

We ran the benchmark suite in machine with an AMD Ryzen 1700 8-core 16-threads @ 3.6Ghz and 16Gb of DDR4 RAM @ 3,200Mhz. As a baseline we decided to run all the tests in our serial implementation of Gale-Shapley. Results are shows in Table \ref{tab:serial-gale-shapley}. In every cell the amount of milliseconds to complete an operation of the algorithm, i.e. get a result, and the margin of error is specified and a confidence interval of 99\%. 

\begin{table}[h]
    \centering
\begin{tabular}{|l|l|l|l|}
\hline
\multicolumn{1}{|c|}{\textbf{n}} & \multicolumn{1}{c|}{\textbf{Best (ms/op)}} & \multicolumn{1}{c|}{\textbf{Random (ms/op)}} & \multicolumn{1}{c|}{\textbf{Worst (ms/op)}} \\ \hline
10                               & 0.011 ±,0.001                              & 0.014 ±,0.001                                & 0.022 ±,0.001                               \\ \hline
100                              & 0.867 ±,0.009                              & 1.198 ±,0.003                                & 6.595 ±,0.227                               \\ \hline
200                              & 3.463 ±,0.064                              & 5.423 ±,0.238                                & 43.587 ±,0.563                              \\ \hline
1000                             & 87.740 ±,0.628                             & 139.563 ±,3.146                              & 8782.987 ± 3636.943                         \\ \hline
\end{tabular}
    \caption{Serial Gale-Shapley}
    \label{tab:serial-gale-shapley}
\end{table}

We then followed by running our parallel version of Gale-Shapley. Results are shown in Table \ref{tab:parallel-gale-shapley}.


\begin{table}[h]
    \centering
\begin{tabular}{|l|l|l|l|}
\hline
\multicolumn{1}{|c|}{\textbf{n}} & \multicolumn{1}{c|}{\textbf{Best (ms/op)}} & \multicolumn{1}{c|}{\textbf{Random (ms/op)}} & \multicolumn{1}{c|}{\textbf{Worst (ms/op)}} \\ \hline
10                               & 0.176 ±,0.011                              & 0.177 ±,0.004                                & 0.176 ±,0.010                               \\ \hline
100                              & 0.379 ±,0.013                              & 0.376 ±,0.011                                & 1.082 ±,0.009                               \\ \hline
200                              & 0.564 ±,0.022                              & 0.728 ±,0.059                                & 5.576 ±,0.106                               \\ \hline
1000                             & 2.115 ±,0.023                              & 3.751 ±,0.168                                & 482.950 ±,31.277                            \\ \hline
\end{tabular}
    \caption{Parallel Gale-Shapley}
    \label{tab:parallel-gale-shapley}
\end{table}

Finally we ran our approach to divide and conquer in parallel. Results are shown in Table \ref{tab:parallel-tseng-lee}

\begin{table}[h]
    \centering
\begin{tabular}{|l|l|l|l|}
\hline
\multicolumn{1}{|c|}{\textbf{n}} & \multicolumn{1}{c|}{\textbf{Best (ms/op)}} & \multicolumn{1}{c|}{\textbf{Random (ms/op)}} & \multicolumn{1}{c|}{\textbf{Worst (ms/op)}} \\ \hline
10                               & 0.168 ±,0.003                              & 0.169 ±,0.007                                & 0.174 ±,0.013                               \\ \hline
100                              & 0.691 ±,0.007                              & 1.008 ±,0.035                                & 1.999 ±,0.025                               \\ \hline
200                              & 2.153 ±,0.034                              & 3.773 ±,0.150                                & 8.559 ±,0.304                               \\ \hline
1000                             & 43.971 ±,0.752                             & 74.536 ±,2.207                               & 401.745 ±,7.819                             \\ \hline
\end{tabular}
    \caption{Parallel Tseng-Lee}
    \label{tab:parallel-tseng-lee}
\end{table}

\begin{comment}
\section{Future Work}
In the effort of having more algorithms to evaluate and compare against the results reported in this paper, implementation of the Parallel Iterative Improvement algorithm can be revisited. Additionally, we believe that using a profiler to check for possible optimizations would also improve our results. The proposed course of action would be to implement this algorithm with CUDA, to make use of a GPUs numerous cores, in an effort to be as close as possible to the $n^2$ processors listed as the ideal resources for the algorithm. Current GPUs would enable realistic results to be obtained for $n$ ranging  up to multiple tens.
To go alone with that effort, finishing debugging and optimizing of the Master-Slave algorithm would be on the list as well (in addition to being the most attainable). This implementation could do with refining and debugging in order to run it against the other implementations. However, because of the bottleneck of the master I expect it to perform worse overall especially in the worst case-scenario.
Having more robust ways to measure effectiveness would help as well. While our benchmarking framework is effective and informative, there are always other things that can be looked. In addition, our implementations may be optimized and allow us to obtain even cleaner results.

\section{Conclusion}
As you can see from the results shown above, we can confidently say that Java Utility Array sort is the fastest and most performant sorting algorithm of all the other algorithms written in Java which should encourage developers to be using it in their applications and not implement their own.

% Temporarily commented out for the proposal
\end{comment}

\bibliographystyle{plain}
\bibliography{references}

\end{document}
